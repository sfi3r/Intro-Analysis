\documentclass[answers]{exam}
\begin{document}
\firstpageheader{}{}{\bf\large Name: Shawn \\ Class: Real Analysis }
\runningheader{Name}{Class Assignment}{Due Date}

\begin{questions}
\question
   \textbf{ Show that if M is the open interval (a,b), and p is in M, then p is a limit point of M.}

\begin{solution}\\
   Let M be the open interval (a,b) and p$\in$M, where a,b$\in$${\rm I\!R}$. \\
  Let S = ($p-\epsilon, p+\epsilon$), where $\epsilon$ is an arbitrary positive real number and $|\epsilon|$ is such that $S \subseteq M$. \\
($p+\epsilon / 2$), p$\in$S for any $\epsilon$$\in$${\rm I\!R}^+$, and so S is an arbitrary interval that contains p and contains another point of M. \\
Therefore p is a limit point of M.
\end{solution}


\question

\textbf{Show that if M is the closed interval [a,b], and p is not in M, then p is not a limit point of M.}

\begin{solution}\\
    Let M be the closed interval [a,b], and p$\notin$M. Assume, without loss of generality, that p $>$ b.\\ 
Let $\epsilon = \frac{p-b}{2}$, and let S = ($p-\epsilon, p+\epsilon$). Hence $S \cap M = \emptyset$. \\ 
Since S is an open interval that contains p but doesn't contain any element of M, p cannot be a limit point of M.
\end{solution}



\question

\textbf{Show that if M is a point set having a limit point, then M contains 2 points}

\begin{solution}\\
   Let M be a point set that contains a limit point \textit{p}. Let $S_\epsilon= (p-\epsilon,p+\epsilon)$, where $\epsilon\in$${\rm I\!R^+}$.
Since p is a limit point there must be exist point in $S_\epsilon$ (besides p) for every $\epsilon$. Let k be another point in $S_\epsilon$, where k$\neq$p.
Now let $S_{\epsilon_2}= (p-\epsilon_2,p+\epsilon_2)$ where $\epsilon_2< |p-k|$. So then $k\notin S_{\epsilon_2}$, but there must exist another point in $S_{\epsilon_2}$
that is not p. \\
Using the Archemedian property and the above argument shows that there are infinite points in M.
\end{solution}

\end{questions}
\end{document}